% ============================================================================
% Chapter 1: Introduction
% ============================================================================
\chapter{Introduction}

\section{Context}

In the current landscape of distributed information systems, log production has become exponential. These logs originate from multiple sources: web applications, microservices, databases, IoT systems, servers, APIs, and more. This massive amount of data constitutes a goldmine of crucial information for several strategic objectives:

\begin{itemize}
    \item \textbf{Incident Detection and Diagnosis:} Quickly identifying anomalies and system failures
    \item \textbf{Performance Analysis:} Optimizing response times and user experience
    \item \textbf{Security and Compliance:} Detecting intrusions and respecting regulations (GDPR, PCI-DSS)
    \item \textbf{Business Intelligence:} Making decisions based on user behavior analysis
\end{itemize}

However, manual management of millions of log lines quickly becomes impossible. It is therefore necessary to implement an automated infrastructure capable of collecting, indexing, searching, and visualizing this data in real-time.

\section{Problem Statement}

Modern enterprises face the following challenges:

\begin{enumerate}
    \item Logs are dispersed across different servers and applications
    \item The lack of centralization makes problem diagnosis long and complex
    \item Technical teams waste time manually searching through log files
    \item There is no automatic alerting system for critical events
    \item Visualization of trends and patterns is non-existent
\end{enumerate}

\section{Project Objectives}

This project aims to design, develop, and deploy a complete log monitoring and analysis platform that addresses enterprise needs. The platform must:

\begin{enumerate}
    \item \textbf{Centralize} log collection from different sources
    \item \textbf{Intelligently index} data for ultra-fast search capabilities
    \item \textbf{Provide relevant visualizations} to facilitate analysis
    \item \textbf{Offer an intuitive web interface} for technical and business teams
    \item \textbf{Be easily deployable and scalable} through containerization
\end{enumerate}

\section{Chosen Scenario: SaaS Web Application}

For this project, we selected \textbf{Scenario D: SaaS Web Application}. This scenario involves developing a monitoring platform for a Software as a Service (SaaS) application used by thousands of client companies.

\subsection{Types of Logs Processed}

\begin{itemize}
    \item \textbf{Web Server Logs:} Apache/Nginx access and error logs
    \item \textbf{Application Logs:} Flask exceptions, warnings, and debug messages
    \item \textbf{Database Logs:} Slow queries, deadlocks, and errors
    \item \textbf{Performance Logs:} Response times, memory/CPU consumption
    \item \textbf{API Logs:} Endpoints called, parameters, response codes
\end{itemize}

\subsection{Key Performance Indicators (KPIs)}

\begin{table}[H]
\centering
\begin{tabular}{ll}
\toprule
\textbf{KPI} & \textbf{Description} \\
\midrule
Error Rate by Service & Percentage of failed requests per service \\
Average Response Time & Mean API response latency \\
Slowest Endpoints & Top endpoints by response time \\
Errors per Hour & Error frequency tracking \\
Active Users & Monthly Active Users (MAU) \\
\bottomrule
\end{tabular}
\caption{Key Performance Indicators for SaaS Monitoring}
\end{table}

\subsection{Priority Use Cases}

\begin{enumerate}
    \item Alert technical teams in case of abnormal error spikes
    \item Identify slow SQL queries to optimize the database
    \item Track API usage to anticipate scaling needs
    \item Generate SLA compliance reports for clients
    \item Provide real-time dashboards for operations teams
\end{enumerate}

\section{Document Structure}

This technical documentation is organized into three main parts:

\begin{description}
    \item[Part 1: Overview and Architecture] Presents the global system architecture and detailed component descriptions
    \item[Part 2: Technical Specifications] Details the technologies, modules, configurations, and data models
    \item[Part 3: Guides and Validation] Provides installation, user guides, testing results, and future perspectives
\end{description}
