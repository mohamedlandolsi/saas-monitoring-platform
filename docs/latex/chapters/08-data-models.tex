% ============================================================================
% Chapter 8: Data Models
% ============================================================================
\chapter{Data Models}

\section{MongoDB Collections}

\subsection{Files Collection}

Stores metadata for uploaded log files.

\begin{lstlisting}[language=json, caption={File Document Schema}]
{
  "_id": ObjectId("507f1f77bcf86cd799439011"),
  "filename": "processed_logs_1704067200.json",
  "original_filename": "server_logs.json",
  "file_type": "json",
  "file_size": 1048576,
  "status": "processed",
  "uploaded_at": ISODate("2024-01-01T10:00:00Z"),
  "processed_at": ISODate("2024-01-01T10:05:00Z"),
  "records_count": 10000,
  "user_id": "user_abc123",
  "error_message": null
}
\end{lstlisting}

\begin{table}[H]
\centering
\begin{tabular}{lll}
\toprule
\textbf{Field} & \textbf{Type} & \textbf{Description} \\
\midrule
\_id & ObjectId & Unique identifier \\
filename & String & Processed filename \\
original\_filename & String & User's original filename \\
file\_type & String & csv or json \\
file\_size & Integer & Size in bytes \\
status & String & pending/processing/processed/error \\
uploaded\_at & Date & Upload timestamp \\
processed\_at & Date & Processing completion \\
records\_count & Integer & Number of log entries \\
user\_id & String & Uploader's user ID \\
\bottomrule
\end{tabular}
\caption{Files Collection Schema}
\end{table}

\subsection{Users Collection}

Stores user authentication and profile data.

\begin{lstlisting}[language=json, caption={User Document Schema}]
{
  "_id": ObjectId("507f1f77bcf86cd799439012"),
  "username": "admin",
  "email": "admin@example.com",
  "password_hash": "$2b$12$...",
  "created_at": ISODate("2024-01-01T00:00:00Z"),
  "last_login": ISODate("2024-01-15T08:30:00Z"),
  "is_active": true,
  "role": "admin"
}
\end{lstlisting}

\subsection{Search History Collection}

Automatically logs all search queries.

\begin{lstlisting}[language=json, caption={Search History Document Schema}]
{
  "_id": ObjectId("507f1f77bcf86cd799439013"),
  "query": "level:ERROR AND endpoint:/api/users",
  "filters": {
    "level": "ERROR",
    "time_range": "last_24h"
  },
  "results_count": 156,
  "execution_time_ms": 45.2,
  "timestamp": ISODate("2024-01-15T10:30:00Z"),
  "user_id": "user_abc123"
}
\end{lstlisting}

\subsection{Saved Searches Collection}

Stores user-configured saved searches.

\begin{lstlisting}[language=json, caption={Saved Search Document Schema}]
{
  "_id": ObjectId("507f1f77bcf86cd799439014"),
  "name": "Production Errors",
  "description": "All errors from production servers",
  "query": "level:ERROR",
  "filters": {
    "server": "prod-*",
    "level": "ERROR"
  },
  "user_id": "user_abc123",
  "created_at": ISODate("2024-01-10T14:00:00Z"),
  "is_public": false
}
\end{lstlisting}

\section{Redis Data Structures}

\subsection{Cache Keys}

\begin{table}[H]
\centering
\begin{tabular}{lll}
\toprule
\textbf{Key Pattern} & \textbf{Type} & \textbf{TTL} \\
\midrule
cache:stats:* & String (JSON) & 60s \\
cache:search:* & String (JSON) & 300s \\
session:* & Hash & 3600s \\
metrics:api:* & Sorted Set & 86400s \\
queue:logs & List & - \\
\bottomrule
\end{tabular}
\caption{Redis Key Patterns}
\end{table}

\subsection{Cache Strategy}

\begin{lstlisting}[language=Python, caption={Redis Caching Pattern}]
def cache_result(timeout=300, key_prefix="cache"):
    """Decorator for caching function results."""
    def decorator(f):
        @wraps(f)
        def wrapper(*args, **kwargs):
            cache_key = f"{key_prefix}:{hash(args)}"
            
            # Try to get from cache
            cached = redis_client.get(cache_key)
            if cached:
                return json.loads(cached)
            
            # Execute and cache result
            result = f(*args, **kwargs)
            redis_client.setex(
                cache_key, 
                timeout, 
                json.dumps(result)
            )
            return result
        return wrapper
    return decorator
\end{lstlisting}

\section{Elasticsearch Index Structure}

\subsection{Document Example}

\begin{lstlisting}[language=json, caption={Elasticsearch Log Document}]
{
  "_index": "saas-logs-2024.01.15",
  "_id": "abc123",
  "_source": {
    "@timestamp": "2024-01-15T10:30:45.123Z",
    "level": "ERROR",
    "endpoint": "/api/users/123",
    "status_code": 500,
    "response_time_ms": 1250.5,
    "client_ip": "192.168.1.100",
    "server": "prod-web-01",
    "message": "Database connection timeout",
    "tags": ["stream", "tcp"]
  }
}
\end{lstlisting}

\subsection{Aggregation Examples}

\begin{lstlisting}[language=json, caption={Error Count by Level Aggregation}]
GET saas-logs-*/_search
{
  "size": 0,
  "aggs": {
    "errors_by_level": {
      "terms": {
        "field": "level.keyword",
        "size": 10
      }
    },
    "avg_response_time": {
      "avg": {
        "field": "response_time_ms"
      }
    },
    "errors_over_time": {
      "date_histogram": {
        "field": "@timestamp",
        "fixed_interval": "1h"
      }
    }
  }
}
\end{lstlisting}
